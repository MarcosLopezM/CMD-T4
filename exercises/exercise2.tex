\documentclass[./../main.tex]{subfiles}
\graphicspath{{img/}}

\begin{document}
    \problempts{5}
    \section{Formulación de deformación plana}
    
    Escribir las siguientes relaciones para la formulación de deformación plana:

    \begin{itemize}
        \item Deformación-desplazamiento
        \item Ecuaciones de equilibrio
        \item Ecuaciones de Hooke
        \item Ecuaciones de Navier
        \item Ecuaciones de Michell-Beltrami
    \end{itemize}

    Para esta formulación tenemos que el tensor de deformaciones es

    \begin{equation}
        \varepsilon_{ij} = 
        \begin{bNiceMatrix}
            \varepsilon_{xx} & \varepsilon_{xy} & 0\\
            \varepsilon_{yx} & \varepsilon_{yy} & 0\\
            0 & 0 & 0.
        \end{bNiceMatrix}
        \label{eq:strain_tensor_deformation}
    \end{equation}

    Y que la relación de deformación-desplazamiento es igual a la de tensión plana \cref{eq:strain_displacement_relationship}, \emph{i.e.},

    \begin{empheq}[box = \resultbox]{equation}
        \varepsilon_{ij} = 
        \begin{bNiceMatrix}
            \pdv{u}{x} & \tfrac{1}{2}\Bigl(\pdv{u}{y} + \pdv{v}{x}\Bigr) & 0\\
            \tfrac{1}{2}\Bigl(\pdv{u}{y} + \pdv{v}{x}\Bigr) & \pdv{v}{y} & 0\\
            0 & 0 & 0    
        \end{bNiceMatrix}.
        \label{eq:strain_displacement_relationship_deformation}
    \end{empheq}

    Mientras que las ecuaciones de equilibrio son iguales a las de tensión plana \cref{eq:equilibrium_equations},

    \begin{empheq}[box=\resultbox]{equation}
        \begin{aligned}
            \pdv{\sigma_{xx}}{x} + \pdv{\sigma_{yx}}{y} + B_{x} &= 0,\\
            \pdv{\sigma_{xy}}{x} + \pdv{\sigma_{yy}}{y} + B_{y} &= 0.
        \end{aligned}
        \label{eq:equilibrium_equations_deformation}
    \end{empheq}

    La ley de Hooke para este caso está dada por

    \begin{equation}
        \sigma_{ij} = \lambda\delta_{ij}\varepsilon_{kk} + 2\mu\varepsilon_{ij},
        \label{eq:hooke_equation_deformation}
    \end{equation}

    \pagebreak
    tal que,

    \begin{align*}
        \sigma_{xx} &= \lambda(\varepsilon_{xx} + \varepsilon_{yy} + \varepsilon_{zz}) + 2\mu\varepsilon_{xx},\\
        \sigma_{yy} &= \lambda(\varepsilon_{xx} + \varepsilon_{yy} + \varepsilon_{zz}) + 2\mu\varepsilon_{yy},\\
        \sigma_{zz} &= \lambda(\varepsilon_{xx} + \varepsilon_{yy} + \varepsilon_{zz}) + 2\mu\varepsilon_{zz},\\
        \sigma_{xy} &= 2\mu\varepsilon_{xy} = \sigma_{yx},\\
        \sigma_{yz} &= 2\mu\varepsilon_{yz} = \sigma_{zy},\\
        \sigma_{xz} &= 2\mu\varepsilon_{xz} = \sigma_{zx}.
    \end{align*}

    Que por \cref{eq:strain_tensor_deformation} se reducen a

    \begin{empheq}[box=\resultbox]{equation}
        \begin{aligned}
            \sigma_{xx} &= \lambda(\varepsilon_{xx} + \varepsilon_{yy}) + 2\mu\varepsilon_{xx},\\
            \sigma_{yy} &= \lambda(\varepsilon_{xx} + \varepsilon_{yy}) + 2\mu\varepsilon_{yy},\\
            \sigma_{zz} &= \lambda(\varepsilon_{xx} + \varepsilon_{yy}),\\
            \sigma_{xy} &= 2\mu\varepsilon_{xy} = \sigma_{yx},\\
            \sigma_{xz} &= \sigma_{yz} = 0.
        \end{aligned}
        \label{eq:hooke_equations_deformation}
    \end{empheq}

    Ahora, para las ecuaciones de Navier primero sustituimos \cref{eq:strain_displacement_relationship_deformation} en \cref{eq:hooke_equations_deformation},

    \begin{align*}
        \sigma_{xx} &= (\lambda + 2\mu)\pdv{u}{x} + \lambda\pdv{v}{y},\\
        \sigma_{yy} &= (\lambda + 2\mu)\pdv{v}{y} + \lambda\pdv{u}{x},\\
        \sigma_{zz} &= \lambda\Bigl(\pdv{u}{x} + \pdv{v}{y}\Bigr),\\
        \sigma_{xy} &= \mu\Bigl(\pdv{u}{y} + \pdv{v}{x}\Bigr).
    \end{align*}

    \pagebreak
    Introduciendo las relaciones anteriores en las ecuaciones de equilibrio \cref{eq:equilibrium_equations_deformation} tenemos que las ecuaciones de Navier son:

    \begin{align}
        \implies\pdv{}{x}\Biggl[(\lambda + 2\mu)\pdv{u}{x} + \lambda\pdv{v}{y}\Biggr] + \pdv{}{y}\Biggl[\mu\Bigl(\pdv{u}{y} + \pdv{v}{x}\Bigr)\Biggr] + B_{x} &= 0,\nonumber\\
        \lambda\pdv[order=2]{u}{x} + \mu\pdv[order=2]{u}{x} + \mu\pdv[order=2]{u}{x} + \lambda\pdv{v}{x,y} + \mu\pdv[order=2]{u}{y} + \mu\pdv{v}{x,y} + B_{x} &= 0,\nonumber\\
        \mu\Biggl(\pdv[order=2]{u}{x} + \pdv[order=2]{u}{y}\Biggr) + (\lambda + \mu)\pdv{}{x}\Biggl(\pdv{u}{x} + \pdv{v}{y}\Biggr) + B_{x} &= 0.\label{eq:navier_equation_x}
    \end{align}

    \begin{align}
        \implies\pdv{}{x}\Biggl[\mu\Bigl(\pdv{u}{y} + \pdv{v}{x}\Bigr)\Biggr] + \pdv{}{y}\Biggl[(\lambda + 2\mu)\pdv{v}{y} + \lambda\pdv{u}{x}\Biggr] + B_{y} &= 0,\nonumber\\
        \mu\Biggl(\pdv[order=2]{v}{x} + \pdv[order=2]{v}{y}\Biggr) + (\lambda + \mu)\pdv{}{y}\Biggl(\pdv{u}{x} + \pdv{v}{y}\Biggr) + B_{y} &= 0.\label{eq:navier_equation_y}
    \end{align}

    Reescribiendo \cref{eq:navier_equation_x,eq:navier_equation_y} en notación vectorial, las ecuaciones de Navier para la deformación plana son

    \begin{empheq}[box=\resultbox]{align*}
        \mu\lap{u} + (\lambda + \mu)\pdv{}{x}\Biggl(\pdv{u}{x} + \pdv{v}{y}\Biggr) + B_{x} &= 0,\\
        \mu\lap{v} + (\lambda + \mu)\pdv{}{y}\Biggl(\pdv{u}{x} + \pdv{v}{y}\Biggr) + B_{y} &= 0.
    \end{empheq}

    Para obtener las ecuaciones de Michell-Beltrami, invertimos las relaciones \cref{eq:hooke_equations_deformation}, \emph{i.e.}, es análogo al caso de tensión plana,

    \begin{align*}
        \varepsilon_{xx} + \varepsilon_{yy} = \dfrac{1}{2(\lambda + \mu)}(\sigma_{xx} + \sigma_{yy}),\\
        \varepsilon_{xx} - \varepsilon_{yy} = \dfrac{1}{2\mu}(\sigma_{xx} - \sigma_{yy}).
    \end{align*}

    Para \(\varepsilon_{xx}\),

    \begin{equation}
        \varepsilon_{xx} = \dfrac{(1 + \nu)}{E}((1 - \nu)\sigma_{xx} - \nu\sigma_{yy}).
        \label{eq:epsilon_xx_deformation}
    \end{equation}

    Para \(\varepsilon_{yy}\),

    \begin{equation}
        \varepsilon_{yy} = \dfrac{(1 + \nu)}{E}((1 - \nu)\sigma_{yy} - \nu\sigma_{xx}).
        \label{eq:epsilon_yy_deformation}
    \end{equation}

    Y, además, por \cref{eq:strain_tensor_deformation} \(\varepsilon_{zz} = 0\). Para \(\varepsilon_{xy}\),

    \begin{equation}
        \varepsilon_{xy} = \dfrac{1 + \nu}{E}\sigma_{xy}.
        \label{eq:epsilon_xy_deformation}
    \end{equation}

    Introduciendo \crefrange{eq:epsilon_xx_deformation}{eq:epsilon_xy_deformation} en las ecuaciones de compatibilidad \cref{eq:compatibility_equation},

    \begin{align*}
        \pdv[order=2]{}{y}\Biggl[\Bigl(\dfrac{1 + \nu}{E}\Bigr)((1 -  \nu)\sigma_{xx} - \nu\sigma_{yy})\Biggr] + \pdv[order=2]{}{x}\Biggl[\Bigl(\dfrac{1 + \nu}{E}\Bigr)((1 -  \nu)\sigma_{yy} - \nu\sigma_{xx})\Biggr] &= 2\Bigl(\dfrac{1 + \nu}{E}\Bigr)\pdv{\sigma_{xy}}{x,y},\\
        \Biggl(\pdv[order=2]{\sigma_{xx}}{x} + \pdv[order=2]{\sigma_{xx}}{y}\Biggr) + \Biggl(\pdv[order=2]{\sigma_{yy}}{x} + \pdv[order=2]{\sigma_{yy}}{y}\Biggr) - \nu\Biggl(\pdv[order=2]{\sigma_{xx}}{x} + \pdv[order=2]{\sigma_{xx}}{y}\Biggr) - \nu\Biggl(\pdv[order=2]{\sigma_{yy}}{x} + \pdv[order=2]{\sigma_{yy}}{y}\Biggr) &= 2\pdv{\sigma_{xy}}{x,y} + \pdv[order=2]{\sigma_{xx}}{x} + \pdv[order=2]{\sigma_{yy}}{y},\\
        (1 - \nu)\lap{(\sigma_{xx} + \sigma_{yy})} &= 2\pdv{\sigma_{xy}}{x,y} + \pdv[order=2]{\sigma_{xx}}{x} + \pdv[order=2]{\sigma_{yy}}{y}.
    \end{align*}

    Sustituyendo \cref{eq:volumen_forces_tension} en la expresión anterior, tenemos que la ecuación de Michell-Beltrami para la formulación de deformación plana es

    \begin{empheq}[box=\resultbox]{equation*}
        \lap{(\sigma_{xx} + \sigma_{yy})} = -\dfrac{1}{1 - \nu}\Biggl(\pdv{B_{x}}{x} + \pdv{By}{y}\Biggr).
    \end{empheq}
\end{document}