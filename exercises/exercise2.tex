\documentclass[./../main.tex]{subfiles}
\graphicspath{{img/}}

\begin{document}
    \problempts{15}
    \section{This is a Problem Worth 15 Points}

    Dibuja un esquema del desdoblamiento total de los niveles de energía de los estados \(n = 2\) y \(n = 3\) del átomo de deuterio debido al efecto de la estructra fina, del efecto Lamb y de la estructura hiperfina. La Figura 1 muestra el esquema correspondiente para el estado \(n = 1\). El spin nuclear del deuterio es \(I = 1\).

    \begin{figure}[htb]
    \centering
    \includegraphics[width=0.8\textwidth, height=0.25\textwidth]{example-image-a}
    \caption{Esquema del desdoblamiento de energía del nivel \(n = 1\) del átomo de deuterio. El esquema no está a escala.}
    \end{figure}

    \textbf{Información útil}:

    \begin{itemize}[label={--}]
    \item Primero, nota que el problema te pide un \textbf{esquema}, es decir, no es necesario que calcules las diferencias de energía exactas de cada nivel. Lo importante es que muestres de manera cualitativa, pero completa, los desdoblamientos finos e hiperfinos del deuterio.
    \item Los niveles de energía de Bohr\footnote{Bohr era genial} son:

    \begin{equation*}
        E_{n} = -\dfrac{1}{2}\mu c^{2}\left(\dfrac{Z\alpha}{n}\right)^{2}.
    \end{equation*}
    \item  La correción a esta energía debida a la estructura fina está dada por:

    \begin{equation*}
        \adif{E_{hf}} = \dfrac{C}{2}[F(F + 1) - I(I + 1) - j(j + 1)],
    \end{equation*}

    en donde

    \begin{equation*}
        C = \dfrac{\mu_{0}}{4\pi}4g_{I}\mu_{B}\mu_{N}\dfrac{1}{j(j + 1)(2\ell + 1)}\left(\dfrac{Z}{a_{\mu}n}\right)^{3}.
    \end{equation*}
    \end{itemize}

    \startsolution[print]

    \kant[1-2]

    \problempts{2.5}
    \section{}

    Muestra que la ecuación de movimiento asociada al desplazamiento \(\fdif{x_{n}}\), de la \(n\)-ésima masa es:
    
    \begin{empheq}[box=\resultbox]{equation}
        m\fdif{\ddot{x}_{n}} = \kappa_{1}(\fdif{x_{n + 1}} - \fdif{x_{n}}) + \kappa_{1}(\fdif{x_{n - 1}} - \fdif{x_{n}}) + \kappa_{2}(\fdif{x_{n + 2}} - \fdif{x_{n}}) + \kappa_{2}(\fdif{x_{n - 2}} - \fdif{x_{n}}).
    \end{empheq}

    \startsolution[print]

            \kant[1-3]
\end{document}